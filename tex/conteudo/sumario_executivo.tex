\chapter*{Sumário Executivo}
\addcontentsline{toc}{chapter}{Sumário Executivo}
\begin{center}
	\parbox{0.7\linewidth}{
		\par O Boletim de Conjuntura do Estado do Tocantins é uma das atividades do Grupo Pet Ciências Econômicas da UFT e tem como objetivo apresentar a evolução das principais variáveis macroeconômicas do estado. Esta edição tem um novo formato com dados trimestrais de 2020, estando a periodicidade das informações   limitada à divulgação de dados pelas fontes oficiais e organizações. Este ano contamos com a parceria do Conselho Regional de Economia (CORECON-TO). As informações contidas são destinadas a cidadãos, gestores públicos e empresários, sendo provenientes de fontes oficiais de organizações públicas.
		
		\par Os textos e as análises apresentados têm caráter   informativo. Os comentários não refletem obrigatoriamente os posicionamentos públicos do CORECON-TO ou da UFT. As análises podem ou não sofrer alterações, caso se confirmem, em função da revisão de dados pelas fontes no que concerne ao   período da análise, a mudanças na conjuntura econômica e social decorrentes de atos governamentais e a forças exógenas, como, por exemplo, o caso da pandemia da COVID-19 este ano. O momento com a pandemia se tornou um desafio para as sociedades brasileira e   mundial.  
		
		\par Neste número, o Boletim traz dados sobre o Produto Interno Bruto (PIB), Orçamento Público, taxa de pobreza, coeficiente de Gini, mercado de trabalho, balança comercial e agricultura. O Produto Interno Bruto corresponde à soma de toda a riqueza de uma nação num determinado período de tempo. Nesta edição, calculamos o PIB pelo lado da demanda e da oferta. Pelo lado da demanda, ele é constituído pela soma do consumo das famílias, governo, investimentos e exportações líquidas; pelo lado da oferta, ele é constituído pela    soma de tudo o que é produzido por todos os setores. Observou-se   retração no primeiro semestre de 2020 na economia brasileira, que se refletiu nos demais estados, inclusive, no Tocantins. 
		
		\par As contas públicas estaduais, Orçamento Público, compreendem as receitas e as despesas do governo. As receitas podem ser provenientes de tributos, transferências, contribuição e de outras fontes, e as despesas, de diferentes setores, como saúde, educação, pessoal, indústria, entre outros. Inclui-se também a capacidade de pagamento do Estado, sua situação fiscal, que compreende endividamento, poupança corrente e liquidez. No campo social, temos a taxa de pobreza e o Índice de Gini. O coeficiente de Gini é uma medida utilizada para calcular a desigualdade na distribuição de renda. Varia entre 0 e 1:   0 significa completa igualdade de renda e 1, completa desigualdade. Por consequência, quanto mais próximo de 1, maior é a concentração de renda. 
		\par A variável Emprego corresponde ao número de pessoas ocupadas formalmente. Apresenta o perfil do empregado (idade, gênero, etnia, grau de instruções), o saldo de emprego do Tocantins e da Região Norte bem como os setores de contratação e demissão, seguro desemprego e rendimento médio. 
		O tópico Balança Comercial traz a evolução dos dados do saldo comercial em dólares de 2009 a 2019. Apresenta os principais produtos exportados e importados e os países    com os quais    o Tocantins tem relação comercial. A agricultura apresenta informações sobre soja, milho e arroz bem como informações sobre a pecuária, em especial, a bovinocultura. 
	}
\end{center}
\thispagestyle{empty}
