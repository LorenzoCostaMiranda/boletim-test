\begin{tcolorbox}[colback=boxbackground, colframe=boxbackground, arc=0mm, top=15pt]
Boletim de Conjuntura do Tocantins é um trabalho realizado pelo Programa de Educação Tutorial (PET) do curso de Ciências Econômicas da Universidade Federal do Tocantins (UFT) em parceria com o Núcleo de Economia Aplicada da UFT.
\\
\par{\bf Equipe}
\begin{itemize}
	\item Coordenação: Prof. Dr. Nilton Marques de Oliveira e Prof. Dr. Marcleiton Ribeiro Morais
	\item Introdução: Lucas Strieder, Gabrielle Dias, Laralisse, Lara Resende
	\item Contas Pública Estadual: Pedro Victor de Sá Castro, Aleksander, Tiago
	\item Empregos: Felipe Ferreira
	\item Balança Comercial: Jean
	\item Indicadores Sociais: Lucas Strieder
	\item Agricultura: Jean, Felipe Ferreira
\end{itemize}
%\par{\bf Convenções estatísticas:}
\par{\bf Dados e Elaboração:}
Este boletim é de acesso livre, seu arquivo em pdf bem como todos os demais arquivos usados na sua elaboração estão disponíveis em um repositório público no endereço \url{https://github.com/peteconomia/boletim}. Comentários e recomendações podem ser enviados para o email: mrm@uft.edu.br.
\\
\par{\bf Informações de Contato}
\begin{itemize}
	\item{Telefone:} (63) XXXX-XXXX
	\item{Email:} mrm@uft.edu.br
	\item{Local:} Universidade Federal do Tocantins (UFT), Palmas, Bloco II, Sala 24. 109 Norte Av. NS-15, ALCNO-14. Plano Diretor Norte. CEP: 77001--090. Av. Juscelino Kubitscheck
\end{itemize}
\par{\bf Direitos de Reprodução:}
É permitida a reprodução do conteúdo desse documento, desde que mencionada a fonte: Boletim de Conjuntura do Tocantins, Brasília v. 1 nº 1 Dez. 2020 p. 1-94.
\end{tcolorbox}


