\begin{tcolorbox}[colback=boxbackground, colframe=boxbackground, arc=0mm, top=15pt]
Boletim de Conjuntura do Tocantins é um trabalho realizado pelo Programa de Educação Tutorial (PET) do curso de Ciências Econômicas da Universidade Federal do Tocantins (UFT).
\\
\par{\bf Equipe:}
\begin{itemize}
	\item{Coordenação:} Prof. Dr. Nilton Marques de Oliveira
	\item{Atividade Econômica:} Lucas Strieder Azevedo, Felipe Ferreira de Sousa, Pedro Victor de Sá, Gabrielle Dias Miranda Santos, Laralisse Carvalho de Oliveira, Lara Resende Castro, Tiago Martins Cirqueira
	\item{Contas Públicas Estadual:} Pedro Victor de Sá Castro, Aleksander Bovo Silva, Tiago Martins Cirqueira
	\item{Indicadores Sociais:} Lucas Strieder Azevedo, Maria Claudia Lemos Oliveira, Daniela Moreira Lopes, Filipe Bastos Romão
	\item{Mercado de Trabalho:} Felipe Ferreira de Sousa, Amanda Vargas Lira, Gabrielle Dias Miranda Santos, Lara Resende Castro
	\item{Comércio Exterior:} Jean Lucas Machado,  Laralisse Carvalho de Oliveira, Heder Soares Azevedo Cordeiro Junior
	\item{Agronegócio:} Felipe Ferreira de Sousa, Jean Lucas Machado, Micauane Oliveira Sousa, Emanuel Pedro Santiago
\end{itemize}
%\par{\bf Convenções estatísticas:}
\par{\bf Dados e Elaboração:}
Este boletim é de acesso livre, seu arquivo em pdf bem como todos os demais arquivos usados na sua elaboração estão disponíveis em um repositório público no endereço \url{https://github.com/peteconomia/boletim}.
\\
\par{\bf Informações de Contato:}
\begin{itemize}
	\item{Telefone:} (63) 3229--4915
	\item{Email:} \url{peteconomia@mail.uft.edu.br}
	\item{Local:} Universidade Federal do Tocantins (UFT), Palmas, Bloco II, Sala 29. 109 Norte Av. NS-15, ALCNO-14. Plano Diretor Norte. CEP: 77001--090. Av. Juscelino Kubitscheck
\end{itemize}
\par{\bf Direitos de Reprodução:}
É permitida a reprodução do conteúdo desse documento, desde que mencionada a fonte: Boletim de Conjuntura do Tocantins, Palmas v. 1 nº 1 Dez. 2020 p. 1--94.
\end{tcolorbox}