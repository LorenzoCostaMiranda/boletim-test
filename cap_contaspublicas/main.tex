\documentclass[twocolumn, a4paper, 10pt]{report}
\usepackage[top=2cm, bottom=2cm, left=2cm, right=2cm]{geometry}
\usepackage[lining]{ebgaramond}
\usepackage[T1]{fontenc}
\usepackage[utf8]{inputenc}
\usepackage[portuguese]{babel}
\usepackage{xcolor}
\usepackage{graphicx}
\usepackage{caption}
\usepackage{subcaption}
\usepackage{tikz}
\usepackage{multicol}
\usepackage{float}
\usepackage{fancyhdr}
\usepackage{titletoc}
\usepackage[explicit]{titlesec}
\usepackage{setspace}
\usepackage{tcolorbox}
\usepackage{pagecolor}
\usepackage{hyperref}
\usepackage{datetime}
\usepackage{kvsetkeys}
\usepackage{lipsum} % generate paragraph
\usepackage{xstring}

% Required by KableExtra R package
%\usepackage{xcolor}
\usepackage{booktabs}
\usepackage{longtable}
\usepackage{array}
\usepackage{multirow}
\usepackage{wrapfig}
%\usepackage{float}
\usepackage{colortbl}
\usepackage{pdflscape}
\usepackage{tabu}
\usepackage{threeparttable}
\usepackage{threeparttablex}
\usepackage[normalem]{ulem}
\usepackage{makecell}

\usepackage[acronym,toc]{glossaries}
\makenoidxglossaries
\IfFileExists{tex/abbreviations.tex}{% Utilize o comando \acrshort{} ou \abbr{} para usar a  sigla. Exemplo \acrshort{pib} ou \abbr{pib}
\newacronym{pib}{PIB}{Produto Interno Bruto}
\newacronym{ibge}{IBGE}{Instituto Brasileiro de Geografia e Estatistica}
\newacronym{bcb}{BCB}{Banco Central do Brasil}
\newacronym{capag}{CAPAG}{Capacidade de Pagamento}
\newacronym{pmc}{PMC}{Pesquisa Mensal do Comércio}
\newacronym{ppc}{PPC}{Paridade do poder de compra}
\newacronym{caged}{CAGED}{Cadastro Geral de Empregados e Desempregados}
\newacronym{cnae}{CNAE}{Classificação Nacional de Atividades Econômicas}
\newacronym{pnadc}{PNAD-C}{Pesquisa Nacional por Amostra de Domicílios Contínua}
\newacronym{mdic}{MDIC}{Ministério da Indústria, Comércio Exterior e Serviços}
\newacronym{fob}{FOB}{Free on Board}
\newacronym{comex}{COMEX STAT}{Estatísticas do Comércio Exterior Brasileiro}
\newacronym{sidra}{SIDRA}{Sistema IBGE de Recuperação Automática}
\newacronym{ipea}{IPEA}{Instituto de Pesquisa Econômica Aplicada}
\newacronym{fieto}{FIETO}{Federação das Indústrias do Estado do Tocantins}
\newacronym{icms}{ICMS}{Imposto sobre Circulação de Mercadorias e Serviços}
\newacronym{rcl}{RCL}{Receita Corrente Líquida}
\newacronym{rca}{RCA}{Receita Corrente Ajustada}
\newacronym{lrf}{LRF}{Lei de Responsabilidade Fiscal}
\newacronym{dcl}{DCL}{Dívida Consolidada Líquida}
\newacronym{siconfi}{SICONFI}{Sistema de Informações Contábeis e Fiscais do Setor Público Brasileiro}
\newacronym{coreconto}{CORECON-TO}{Conselho Regional de Economia do Tocantins}
\newacronym{uft}{UFT}{Universidade Federal do Tocantins}
\newacronym{pet}{PET}{Programa de Educação Tutorial}
\newacronym{mte}{MTE}{Ministério do Trabalho e Emprego}
\newacronym{rreo}{RREO}{Relatório Resumido da Execução Orçamentária}}{% Utilize o comando \acrshort{} ou \abbr{} para usar a  sigla. Exemplo \acrshort{pib} ou \abbr{pib}
\newacronym{pib}{PIB}{Produto Interno Bruto}
\newacronym{ibge}{IBGE}{Instituto Brasileiro de Geografia e Estatistica}
\newacronym{bcb}{BCB}{Banco Central do Brasil}
\newacronym{capag}{CAPAG}{Capacidade de Pagamento}
\newacronym{pmc}{PMC}{Pesquisa Mensal do Comércio}
\newacronym{ppc}{PPC}{Paridade do poder de compra}
\newacronym{caged}{CAGED}{Cadastro Geral de Empregados e Desempregados}
\newacronym{cnae}{CNAE}{Classificação Nacional de Atividades Econômicas}
\newacronym{pnadc}{PNAD-C}{Pesquisa Nacional por Amostra de Domicílios Contínua}
\newacronym{mdic}{MDIC}{Ministério da Indústria, Comércio Exterior e Serviços}
\newacronym{fob}{FOB}{Free on Board}
\newacronym{comex}{COMEX STAT}{Estatísticas do Comércio Exterior Brasileiro}
\newacronym{sidra}{SIDRA}{Sistema IBGE de Recuperação Automática}
\newacronym{ipea}{IPEA}{Instituto de Pesquisa Econômica Aplicada}
\newacronym{fieto}{FIETO}{Federação das Indústrias do Estado do Tocantins}
\newacronym{icms}{ICMS}{Imposto sobre Circulação de Mercadorias e Serviços}
\newacronym{rcl}{RCL}{Receita Corrente Líquida}
\newacronym{rca}{RCA}{Receita Corrente Ajustada}
\newacronym{lrf}{LRF}{Lei de Responsabilidade Fiscal}
\newacronym{dcl}{DCL}{Dívida Consolidada Líquida}
\newacronym{siconfi}{SICONFI}{Sistema de Informações Contábeis e Fiscais do Setor Público Brasileiro}
\newacronym{coreconto}{CORECON-TO}{Conselho Regional de Economia do Tocantins}
\newacronym{uft}{UFT}{Universidade Federal do Tocantins}
\newacronym{pet}{PET}{Programa de Educação Tutorial}
\newacronym{mte}{MTE}{Ministério do Trabalho e Emprego}
\newacronym{rreo}{RREO}{Relatório Resumido da Execução Orçamentária}}

% Cores
\definecolor{primarycolor}{RGB}{0, 96, 157}
\definecolor{secondarycolor}{RGB}{34, 192, 221}
\definecolor{boxbackground}{RGB}{225, 233, 246}
\definecolor{primarytext}{RGB}{0, 0, 0}
\definecolor{secondarytext}{RGB}{180, 180, 180}

% Text Layout
\setlength\parindent{10pt} % Tamanho da indentação do paragrafo
\parskip = 1pt
\setlength{\columnsep}{15pt} % Espaço entre as colunas
\setstretch{1} % Altura da linha
\setcounter{tocdepth}{0} % Table of contents depth, imprime apenas chapter e section

% Figures Caption Config
\captionsetup{
	format=plain,
	justification=raggedright,
	singlelinecheck=false,
	font={normalsize,color=primarycolor},
	labelfont={color=primarycolor},
	labelsep=space,
	skip=0pt
}
\renewcommand{\thesubfigure}{.\arabic{subfigure}}
\DeclareCaptionLabelFormat{opening}{Figura~\thechapter.\arabic{figure}.\arabic{subfigure}}
\captionsetup[subfigure]{
	labelformat=opening,
	font={normalsize,color=primarycolor},
	labelsep=space,
	skip=2pt
}
\setkeys{Gin}{width=\linewidth} % \includegraphics por padrão terá comprimento igual a \linewidth
% Links Config
\hypersetup{
	colorlinks=true,
	linkbordercolor=white,
	linkcolor=primarytext,
	urlcolor=primarytext,
	urlbordercolor=white, %links externos
	citecolor=primarytext,
	pdftitle={},
	pdfauthor={PET - Ciências Econômicas, Universidade Federal do Tocantins},
	pdfsubject={},
	pdfcreator={LaTeX},
	pdfproducer={PET - Ciências Econômicas},
	pdfkeywords={Tocantins, Economia, boletim},
	bookmarks=true
}

% Numeração da Pagina
% Clear the header and footer
\fancyhf{} % clear all
\fancyhead{}
\fancyfoot{}
\fancypagestyle{plain}{
	\renewcommand{\headrulewidth}{0pt}
	\renewcommand{\footrulewidth}{0pt}
	\fancyfoot[RE,RO]{\footnotesize\textcolor{secondarytext}{\leftmark}\quad\thepage}
}
\pagestyle{plain}

% Table Of Contents Style
\titlecontents{chapter}[70pt]
	{\LARGE\color{primarycolor}\bigskip}
	{\thecontentslabel.~}
	{}
	{\enspace---\enspace\contentspage}

\titlecontents{section}[100pt]
	{\large\color{primarycolor}\bigskip}
	{}
	{}
	{\enspace---\contentspage}

\titlecontents{subsection}[130pt]
	{\normalsize\color{primarycolor}\bigskip}
	{}
	{}
	{\enspace---\contentspage}


% Style Chapter, section and subsection
\titleformat{\chapter}[display]
	{\filright}
	{\scriptsize\color{secondarytext}\MakeUppercase\chaptertitlename~\thechapter}
	{5pt} % margem superior
	{\fontsize{50pt}{50pt}\selectfont\color{primarycolor}#1}\titlespacing*{\chapter}
	{0pt}{0pt}{20pt}  %controls vertical margins on title

\titleformat{\section}
	{\large\filright\color{primarycolor}}
	{}
	{0pt}
	{#1}

\titleformat{\subsection}
	{\large\filright\color{primarycolor}}
	{}
	{0pt}
	{#1}

% Pandoc output require
\providecommand{\tightlist}{%
	\setlength{\itemsep}{0pt}\setlength{\parskip}{0pt}}

% Box Config
\newtcolorbox[auto counter,number within=chapter]{smbox}[2][]{
	%float=p,
	colback=boxbackground,
	colframe=boxbackground,
	arc=0mm,
	valign=center,
	top=0pt,
	left=10pt,
	right=10pt,
	bottom=10pt,
	toptitle=20pt,
	bottomtitle=0pt,
	width=\linewidth,
	fonttitle=\color{primarycolor},
	title=Quadro~\thetcbcounter~#2,#1
}

\newcommand{\source}[1]{\scriptsize{Fonte: #1}\\}

\newcommand{\notes}[1]{{\scriptsize{Nota:#1}}}

\newcommand{\subcap}[1]{{\scriptsize\color{primarycolor}#1\newline}}

\newcommand{\abbr}[1]{\acrshort{#1}}

\newcommand{\trimestres}[1][1-4]{
	\IfEqCase{#1}{%
        {1-4}{1T: 1º trimestre, 2T: 2º trimestre, 3T: 3º trimestre, 4T: 4º trimestre}%
		{1-3}{1T: 1º trimestre, 2T: 2º trimestre, 3T: 3º trimestre}%
		{1-2}{1T: 1º trimestre, 2T: 2º trimestre}
		{2-3}{2T: 2º trimestre, 3T: 3º trimestre}
		{2-4}{2T: 2º trimestre, 3T: 3º trimestre, 4T: 4º trimestre}
		{3-4}{3T: 3º trimestre, 4T: 4º trimestre}
		{1}{1T: 1º trimestre}
		{2}{2T: 2º trimestre}
		{3}{3T: 3º trimestre}
		{4}{4T: 4º trimestre}
	}[]
}
\newcommand{\bimestres}[1][1-4]{
	\IfEqCase{#1}{%
        {1-4}{1B: 1º bimestre, 2B: 2º bimestre, 3B: 3º bimestre, 4B: 4º bimestre}%
		{1-3}{1B: 1º bimestre, 2B: 2º bimestre, 3B: 3º bimestre}%
		{1-2}{1B: 1º bimestre, 2B: 2º bimestre}
		{2-3}{2B: 2º bimestre, 3B: 3º bimestre}
		{2-4}{2B: 2º bimestre, 4B: 3º bimestre, 4B: 4º bimestre}
		{3-4}{3B: 3º bimestre, 4B: 4º bimestre}
		{1}{1B: 1º bimestre}
		{2}{2B: 2º bimestre}
		{3}{3B: 3º bimestre}
		{4}{4B: 4º bimestre}
		{5}{5B: 5º bimestre}
		{6}{6B: 6º bimestre}
    }[]
}

\begin{document}
    
    \hypertarget{contas-puxfablicas-estadual}{%
    \chapter{Contas Públicas
    Estadual}\label{contas-puxfablicas-estadual}}

    O resultado primário do estado em 2020 foi de cerca de R\$ 1,45
    bilhões, valor 18,8\% maior que o resultado primário de 2019, quando
    foi pouco mais de R\$ 1,22 bilhões. Veja o Quadro {[}box:
    @resultado\_primario{]} para mais detalhes sobre o resultado
    primário.

    \begin{figure}[!h]
    \begin{subfigure}{\linewidth}
    \caption{Variação da receita e despesa primária\label{fig:variacao_receita_despesa_primaria}}
    \subcap{Variação por bimestre}
    \includegraphics{main_files/figure-latex/variacao_receita_despesa_primaria-1.pdf}
    \source{\abbr{siconfi}}
    \end{subfigure}
    \end{figure}

    As receitas primárias cresceu \% no quinto bimestre de 2020, como
    mostra a Figura. As despesas primárias cresceu 2,03\%. No quarto
    bimestre de 2019 as receitas tinham crescido 9,55\% e as despesas
    6,48\%. Comparando o crescimento das despesas primárias no quarto
    bimestre de 2020 a taxa de crescimento foi menor que em 2019. O
    baixo crescimento da despesas contribuiu para um superávit primário
    de pouco mais de R\$ 1,08 bilhões até o quarto bimestre de 2020.

    A Figura exibe as despesas por categorias. Destaque para as despesas
    com assistência social, que cresceu cerca de 133\% no quarto
    bimestre de 2020. Previdência social, saúde e judiciário cresceu
    16,2\%, 12,9\% e 18,8\% respectivamente. Por outro lado,
    administração, segurança pública e educação recuaram.\\

    \begin{smbox}[label={box:resultado_primario}]{O que é o resultado primário}
    O resultado primário é um dos principais indicadores das contas
    públicas, representa o esforço fiscal para diminuir o estoque da
    dívida. Ele é resultado da diferença entre as receitas e despesas
    (excluindo as receitas e despesas com juros). O superávit primário
    ou resultado primário positivo ocorre quandos as receitas primárias
    é maior que as despesas primárias. Indica a economia do governo para
    pagamento da dívida. O inverso, quando despesas primárias excedem as
    receitas primárias há déficit primário ou resultado primário
    negativo, incorrendo em aumento da dívida.

    \end{smbox}

    Despesas com pessoal em relação a receita corrente líquida
    (\textbackslash abbr\{rcl\}), conforme Figura
    \textbackslash ref\{fig:desp\_pessoal\_rcl\}, encontra-se em 42,1\%
    em agosto de 2020, valor abaixo do limite máximo de 49\%
    estabelecido na Lei de Responsabilidade Fiscal
    (\textbackslash abbr\{lrf\}) para o poder Executivo
    \textbackslash footnote\{A \textbackslash abbr\{rcl\}, de acordo com
    a \textbackslash abbr\{lrf\}, deve ser apurada somando-se as
    receitas arrecadadas no mês em referência e nos onze anteriores. No
    entanto, pelo fato dessa publicação cobrir dados até cerca do
    primero semestre optou-se pela utilização da
    \textbackslash abbr\{rcl\} acumulada até o respectivo bimestre\}. Em
    agosto de 2015 a \textbackslash abbr\{rcl\} destinada ao pagamento
    de pessoal correspondia a 51,5\%, valor acima do limite máximo. O
    comprometimento da \textbackslash abbr\{rcl\} ao pagamento de
    pessoal extrapolou o limite em 2015, 2016, 2017 e 2018.

    A dívida consolidada líquida (\textbackslash abbr\{dcl\}) do estado
    em proporção a \textbackslash abbr\{rcl\} até agosto apresentou
    queda. Em agosto de 2020 essa indicador ficou em 44,1\%, valor
    abaixo do limite definido pelo Senado Federal para os estados, de
    duas vezes a \textbackslash abbr\{rcl\}. Entre 2017 e 2018 a
    \textbackslash abbr\{dcl\} em proporção à \textbackslash abbr\{rcl\}
    aumentou, saindo de 30\textbackslash\% para 52,3\% em 2019, conforme
    Figura \textbackslash ref\{fig:divida\_rcl\}.

    O indicador da capacidade de pagamento
    (\textbackslash abbr\{capag\}) do estado traz informações a cerca da
    situação fiscal dos estados e municípios. O índice é composto por
    três componentes: endividamento, poupança corrente e liquidez.
    Estados e municípios recebem uma nota final, A, B, C ou D.

    O Tocantins ficou com nota C em 2019 e 2020. Mesmo mantendo a mesma
    nota entre 2019--2020, apresentou pioras em todos os indicadores. O
    envididamento do estado que representa a \textbackslash abbr\{dcl\}
    em proporção à \textbackslash abbr\{rcl\} saltou de 46,35\% para
    67,6\%. A poupança corrente que corresponde despesas corrente e
    receitas correntes ajustadas (\textbackslash abbr\{rca\}) também
    mostrou uma leve piora, saindo de 94,56\% para 95,9\%. A liquidez do
    estado cresceu de 539,4\% para 577,5\% em 2020.

    Endividamento e poupança corrente estão em melhor condição, pois
    estão mais próximo do limite para receber uma melhor nota. Para
    obter uma nota A no índice de endividamento o estado deve
    conservá-lo abaixo de 60\%, atualmente está em 67,6\%. A poupança
    corrente recebeu nota C em 2020 conforme Tabela
    \textbackslash ref\{tab:capag\_uf\}. Uma elevação na nota da
    poupança corrente para B requer uma relação despesas correntes e
    \textbackslash abbr\{rca\} menor que 95\%, em 2020 ficou em 95,85\%.
    A liquidez do estado encontra-se em situação mais delicada, em 2020
    fechou em 577,5\%, valor quase cinco vezes acima do limite para
    tirar nota A.

    Dentre os estados da região Norte, Tocantins e Roraima foram os que
    apresentaram pior desempenho, conforme disposto na Tabela
    \textbackslash ref\{tab:capag\_uf\}. Rondônia aparece com a melhor
    perfomance, saiu da nota B para A entre 2019--2020. A redução no
    endividamento e na liquidez garantiu nota A em todos os indicadores.

    \end{document}